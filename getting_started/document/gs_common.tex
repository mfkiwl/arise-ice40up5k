% !TEX encoding = UTF-8 Unicode
\newpage
\section{Pinout information}
The board has multiple I/O pins, as well as power supply pins, distributed along the side and the front of the PCB. The next list gives a short description of the signals:
\begin{itemize}
	\item IOT\_XX are general I/O pins. In user mode, after configuration, these pins can be programmed as I/O in user function in the top (xx = I/O location)
	\item IOB\_XX are general I/O pins. In user mode, after configuration, these pins can be programmed as I/O in user function in the bottom (xx = I/O location)
	\item RAW VCC. The input voltage to the FPGA board when it's using an external power source. The board can be supplied with power either from the USB connector or the RAW VCC pin. The nominal supply voltage is 5V. (Minimum is 4.5V and maximum is 6V)
	\item 3.3V. A 3.3 volt supply generated by the on-board regulator. Maximum recommended current draw from this pin is 500 mA
	\item GND. Ground pins
	\item Signal names with G1, G3 and G6 suffixes may be used as a General I/O or as a Global input used for high fanout, or clock/reset net. These pins drive the GBUF1, GBUF3 and GBUF6 global buffers respectively
\end{itemize}

\newpage
Table \ref{table1} shows the mapping between the signals and the board pin numbers.
\definecolor{myHdr}{rgb}{0.81000,0.8800,0.9400}%
\begin{table}[h!]
	\renewcommand{\arraystretch}{1.3}
	\caption{Connections}
	\label{table1}
	\begin{tabular}[t]{C{0.12\textwidth} C{0.12\textwidth} C{0.15\textwidth}}
		\bfseries Board Pin & \bfseries FPGA Pin & \bfseries Signal Name \\ \hline
		\rowcolor{myHdr} \multicolumn{3}{c}{\bfseries Side Pins} \\ \hline
		1 & 20 & IOB 25B G3 \\
		2 & 21 & IOB 23B \\
		3 & 23 & IOT 37A \\
		4 & 25 & IOT 36B \\
		5 & 26 & IOT 39A \\
		6 & 27 & IOT 38B \\
		7 & 31 & IOT 42B \\
		8 & 32 & IOT 43A \\
		9 & 34 & IOT 44B \\
		10 & 36 & IOT 48B \\
		11 & 37 & IOT 45A G1 \\
		12 & - & GND \\
		& & \\
		13 & 2 & IOB 6A \\
		14 & 6 & IOB 13B \\
		15 & 9 & IOB 16A \\
		16 & 10 & IOB 18A \\
		17 & 11 & IOB 20A \\
		18 & 12 & IOB 22A \\
		19 & 13 & IOB 24A \\
		20 & 18 & IOB 31B \\
		21 & 19 & IOB 29B \\
		22 & - & 3.3V \\
		23 & - & GND \\
		24 & - & RAW VCC \\
	\end{tabular}
	\quad
	\begin{tabular}[t]{C{0.12\textwidth} C{0.12\textwidth} C{0.15\textwidth}}
		\bfseries Board Pin & \bfseries FPGA Pin & \bfseries Signal Name \\ \hline
		\rowcolor{myHdr} \multicolumn{3}{c}{\bfseries Front Pins} \\ \hline
		1 & 4 & IOB 8A \\
		2 & 3 & IOB 9B \\
		3 & 47 & IOB 2A \\
		4 & 44 & IOB 3B G6 \\
		5 & - & GND \\
		6 & - & 3.3V \\
		7 & 48 & IOB 4A \\
		8 & 45 & IOB 5B \\
		9 & 38 & IOT 50B \\
		10 & 42 & IOT 51A \\
		11 & - & GND \\
		12 & - & 3.3V \\
		& & \\
		\bfseries Board Pin & \bfseries FPGA Pin & \bfseries Signal Name \\ \hline
		\rowcolor{myHdr} \multicolumn{3}{c}{\bfseries Not Connected} \\ \hline
		 - & 43 & IOT 49A \\
		- & 46 & IOB 0A \\
		- & 28 & IOT 41A \\	
		& & \\
		& & \\
		\bfseries LED Color & \bfseries FPGA Pin & \bfseries Signal Name \\ \hline
		\rowcolor{myHdr} \multicolumn{3}{c}{\bfseries Onboard RGB LED} \\   \hline
		Blue & 39 & RGB0 \\
		Green & 40 & RGB1 \\
		Red & 41 & RGB2 \\
	\end{tabular}
\end{table}

\newpage
The differential pairs are shown in Table \ref{table2}. They are grouped together and labeled in colors, white is positive and light blue is negative.
%
\begin{table}[h!]
	\renewcommand{\arraystretch}{1.3}
	\caption{Differential Pairs}
	\vspace{0.5em}
	\label{table2}
	\centering
	\begin{tabular}{C{0.12\textwidth} C{0.12\textwidth} C{0.15\textwidth}}
		\bfseries Board Pin & \bfseries FPGA Pin & \bfseries Signal Name \\ \hline
		\rowcolor{myHdr}  3 & 23 & IOT 37A \\
		4 & 25 & IOT 36B \\ \hline
		\rowcolor{myHdr}  5 & 26 & IOT 39A \\
		6 & 27 & IOT 38B \\ \hline
		\rowcolor{myHdr}  8 & 32 & IOT 43A \\
		7 & 31 & IOT 42B \\ \hline
		\rowcolor{myHdr}  11 & 37 & IOT 45A G1 \\
		9 & 34 & IOT 44B \\ \hline
		\rowcolor{myHdr}  2 & 21 & IOB 23B \\
		18 & 12 & IOB 22A \\ \hline
		\rowcolor{myHdr}  2 & 3 & IOB 9B \\
		1 & 4 & IOB 8A \\ \hline
		\rowcolor{myHdr}  4 & 44 & IOB 3B G6 \\
		3 & 47 & IOB 2A \\ \hline
		\rowcolor{myHdr}  8 & 45 & IOB 5B \\ 
		7 & 48 & IOB 4A \\ \hline
		\rowcolor{myHdr}  10 & 42 & IOT 51A \\ 
		9 & 38 & IOT 50B \\
	\end{tabular}
\end{table}

\section{Open Source PCB}
The board files, circuit schematics and list of components are available in the GitHub repository. Altium project and Gerber files are available. The board is also published in open source EDAs just as \href{https://workspace.circuitmaker.com/}{Circuit Maker} and \href{http://www.kicad-pcb.org}{KiCad}.

\newpage
\subsection{Bill of Materials}
Table \ref{table3} includes all the components in the FPGA Board. Follow the hypertext links to the component's datasheets for further information.
%
\begin{table}[h]
	\renewcommand{\arraystretch}{1.3}
	\caption{Bill of materials}
	\vspace{0.5em}
	\label{table3}
	\centering
	\begin{tabular}{p{4cm} C{0.5cm} c C{1cm} p{2cm}}
		\bfseries Designator & \bfseries Qty & \bfseries Manufacturer Part Number & \bfseries Value & \bfseries Footprint \\ \hline
		C2, C3, C4, C5, C6, C10, C12, C14, C16, C18, C19, C21, C23 & 13 & \href{http://www.samsungsem.com/kr/support/product-search/mlcc/CL05B104KA5NNNC.jsp}{CL05B104KA5NNNC} & 0.1uF & 0402 \\ \hline
		C8, C9 & 2 & \href{http://www.yageo.com/documents/recent/UPY-GPHC_X5R_4V-to-50V_25.pdf}{CC0603KRX5R8BB105} & 1uF & 0603 \\ \hline
		C1, C7, C11, C13, C15, C17, C20, C22 & 8 & \href{https://product.tdk.com/info/en/catalog/datasheets/mlcc_commercial_lowprofile_en.pdf}{CGB2A1JB1E105M033BC} & 1uF & 0402 \\ \hline
		R1 & 1 & \href{http://www.yageo.com/documents/recent/PYu-RC_Group_51_RoHS_L_9.pdf}{RC0402JR-071KL} & 1K$\Omega$ & 0402 \\ \hline
		R2 & 1x& \href{http://www.yageo.com/documents/recent/PYu-RC_Group_51_RoHS_L_9.pdf}{RC0402FR-072K2L} & 2.2k$\Omega$ & 0402 \\ \hline
		R3, R4, R5, R6, R7, R13 & 6 & \href{http://www.yageo.com/NewPortal/yageodocoutput?fileName=/pdf/R-Chip/PYu-RC_51_RoHS_P_0.pdf}{RC0402JR-0710KP} & 10k$\Omega$ & 0402 \\ \hline
		R8, R9, R10, R11, R12 & 5 & \href{http://www.yageo.com/NewPortal/yageodocoutput?fileName=/pdf/R-Chip/PYu-AC_51_RoHS_L_6.pdf}{AC0402JR-071RL} & 1$\Omega$ & 0402 \\ \hline
		L1 & 1 & \href{http://assets.lairdtech.com/home/brandworld/files/HI0603P600R-10.pdf}{HI0603P600R-10} & 10mH & 0603 \\ \hline
		L2, L3, L4 & 3 & \href{https://www.murata.com/en-us/products/productdata/8796741599262/ENFA0004.pdf}{BLM18HE601SN1D} & 10mH & 0603 \\ \hline
		RGB & 1 & \href{https://www.cree.com/led-components/media/documents/1273-CLMVC-FKA.pdf}{CLMVC-FKA-CL1D1L71BB7C3C3} & & 4-PLCC \\ \hline
		D1 & 1 & \href{http://optoelectronics.liteon.com/upload/download/DS-22-99-0224/LTST-C190TBKT.PDF}{LTST-C190TBKT} & & 0603 \\ \hline
		U1 & 1 & \href{http://www.ftdichip.com/Support/Documents/DataSheets/ICs/DS_FT232H.pdf}{FT232HL-REEL} & & 48-LQFP (7x7) \\ \hline    
		U2 & 1 & \href{http://www.ti.com/lit/ds/symlink/tlv1117lv.pdf}{TLV1117LV33DCYR} & & SOT-223-4 \\ \hline
		U3 & 1 & \href{http://www.ti.com/lit/ds/symlink/lp5907.pdf}{LP5907MFX-1.2/NOPB} & & SOT-23-5 \\ \hline
		OSC1 & 1 & \href{https://media.digikey.com/pdf/Data\%20Sheets/SiTime\%20PDFs/SIT1602A.pdf}{SIT1602AC-73-33S-12.000000G} & & 2.0X1-6MM \\ \hline
		FPGA & 1 & \href{http://www.latticesemi.com/view_document?document_id=51968}{ICE40UP5K-SG48ITR50} & & 48-QFN-7X7 \\ \hline
		USB & 1 & \href{http://www.amphenol-icc.com/media/wysiwyg/files/drawing/10118193.pdf}{10118193-0001LF} & & Micro USB B SMD \\ \hline
		MEM & 1 & \href{https://www.winbond.com/resource-files/w25q32jv%20dtr%20revf%2002242017.pdf}{W25Q32JVSSIQ} & & SOP8 \\ \hline
		D2, D3, D4 & 3 & \href{http://www.comchiptech.com/admin/files/product/CDBU0520-HF-RevA797161.pdf}{CDBU0520} & & 0603/SOD-523F \\ \hline
		SW1 & 1 & \href{https://www.ckswitches.com/media/1465/kxt3.pdf}{PTS810 SJM 250 SMTR LFS} & & SW4-SMD \\
	\end{tabular}
\end{table}

\subsection{Schematics}
\includepdf[landscape=true]{figs/ftdi.pdf}
\includepdf[landscape=true]{figs/misc.pdf}
\includepdf[landscape=true]{figs/fpga.pdf}

\section{License}
This project is licensed under a Creative Commons Attribution Share-Alike license, meaning that you are free to use and adapt it for your own needs without asking for permission or paying a fee, even for commercial purposes, as long as you give appropriate credit and release the design under the same license. Visit the
\href{https://creativecommons.org/licenses/by-sa/3.0/}{Creative Commons Website} for further information.

\section{Acknowledgments}
The circuit was inspired by the official manuals and toolchain of the \href{http://www.latticesemi.com/en/Products/DevelopmentBoardsAndKits/iCE40UltraPlusBreakoutBoard.aspx}{iCE40 UltraPlus Breakout Board} from Lattice, and the documentation of the iCE40UP5K FPGA chip.
\documentclass{beamer}
\usetheme{Madrid}
\usecolortheme{default}

\definecolor{BlueITBA}{HTML}{00558c} 
\colorlet{beamer@blendedblue}{BlueITBA}
\setbeamertemplate{caption}{\scriptsize\insertcaption}
\setbeamertemplate{itemize items}[circle]
%\setbeamertemplate{itemize subitem}[triangle]
\setbeamertemplate{itemize subitem}{\raisebox{0.2em}{\scalebox{0.65}{$\blacktriangleright$}}}   % Second level

%\setbeamertemplate{enumerate items}[default]
\setbeamertemplate{section in toc}{\inserttocsectionnumber.~\inserttocsection}

\usepackage{sansmathfonts}				% Sans Serif equations
\usepackage[T1]{fontenc}					% Output font encoding for international characters
\usepackage[utf8]{inputenc}				% Encoding of files: utf8
\renewcommand*\familydefault{\sfdefault} 		% Sans Serif as default font
\usepackage{verbatim}
 
% Setup the language and its properties (choose only one)
\usepackage[spanish, es-tabla, es-nodecimaldot]{babel}
\addto\captionsspanish{\renewcommand{\contentsname}{Contenido}}

\usepackage{ulem}
\usepackage{graphicx}
\usepackage{tikz}						% Required for title page
\usepackage{hyperref}
\hypersetup{
    colorlinks=true,
    linkcolor=white,
    filecolor=magenta,      
    urlcolor=blue,
    citecolor=blue,    
}
\usepackage{listings}

\title[Primeros Pasos]{Primeros Pasos Radiant}
\subtitle{}
\author{Dr. Ing. Pablo \textsc{Cossutta}}
%\institute[]{Dr. Ing. Miguel \textsc{Aguirre} - Ing. Analía \textsc{Douthat}}
\date{Octubre 2019}
\logo{\includegraphics[height=1.5cm]{../../logo_itba.png}} 
 
\begin{document}
\beamertemplatenavigationsymbolsempty % Disable Nav Bar
% Title and Intro
\maketitle
\logo{}

\begin{frame}{Tutorial}
¿Qué se va a hacer?
\begin{itemize}
	\item Instalar Radiant
	\begin{itemize}
		\item Requiere registrarse
		\item Requiere licencia
	\end{itemize}
	\item Crear un proyecto
	\item Cargar los archivos en el proyecto
	\item Synthesis, Map, Place \& Route, Bitstream
	\item Cargar el diseño
\end{itemize}
¿Qué se necesita?
\begin{itemize}
	\item Archivo HDL (top.v)
	\item Constraints para síntesis (impl\_1.ldc)
	\item Constraints para implementación (impl\_1.pdc)	
\end{itemize}
\end{frame}
    
\begin{frame}[fragile]{top.v - Definition and clock divider}
\begin{lstlisting}[language=Verilog]
module top(
   output RGB0,
   output RGB1,
   output RGB2,
   input clk
);

   reg [21:0] div = 22'd0;
   always @(posedge clk) div <= div + 1'd1;

   reg en;
   always @(posedge clk) begin
      if (div==22'd0) en <= 1'b1;
      else en <= 1'b0;
   end
\end{lstlisting}
\end{frame}

\begin{frame}[fragile]{top.v - Decoding}
\begin{lstlisting}[language=Verilog]
   reg [2:0] rgb_reg = 3'd0;
   always @(posedge clk) begin
      if (en) begin
         case (rgb_reg)
            3'b000:  rgb_reg <= 3'b001;
            3'b001:  rgb_reg <= 3'b010;
            3'b010:  rgb_reg <= 3'b100;
            3'b100:  rgb_reg <= 3'b000;
            default: rgb_reg <= 3'b000;
         endcase
      end
   end
\end{lstlisting}
\end{frame}

\begin{frame}[fragile]{top.v - RGB Instantiation}
\begin{lstlisting}[language=Verilog]
   RGB #(
     .CURRENT_MODE ("1"),
     .RGB0_CURRENT ("0b000001"),
     .RGB1_CURRENT ("0b000001"),
     .RGB2_CURRENT ("0b000001")
   ) RGB_INST (
     .CURREN   (1'b1),        // I
     .RGBLEDEN (1'b1),        // I
     .RGB0PWM  (rgb_reg[0]),  // I - Blue
     .RGB1PWM  (rgb_reg[1]),  // I - Green
     .RGB2PWM  (rgb_reg[2]),  // I - Red
     .RGB0     (RGB0),        // O
     .RGB1     (RGB1),	   // O
     .RGB2     (RGB2)         // O
   );
endmodule
\end{lstlisting}
\end{frame}

\begin{frame}[fragile]{Restricciones (Constraints)}
Restricciones lógicas
\begin{itemize}
	\item Generalmente necesarias para la síntesis
\end{itemize}
\begin{block}{impl\_1.ldc}
\begin{lstlisting}
create_clock -name {clk} -period 83.333 
-waveform \{0.000 41.666\} [get_ports clk]
\end{lstlisting}
\end{block}
\vfill
Restricciones físicas
\begin{itemize}
	\item Necesarias para la implementación
\end{itemize}
\begin{block}{impl\_1.pdc}
\begin{lstlisting}
ldc_set_location -site {35} [get_ports clk]
\end{lstlisting}
\end{block}
\end{frame}

\begin{frame}[fragile]{Trabajo en clase (Hands On)}
Utilizando los conocimientos adquiridos y la documentación correspondiente a la plataforma, implementar el siguiente diseño:
\begin{lstlisting}[language=Verilog]
module top(
   output j05,
   input j01, j02, j03, j04);

   assign j05 = j01 & j02 & j03 & j04;
endmodule
\end{lstlisting}
Cargar en la FPGA y analizar:
\begin{itemize}
	\item Analizar RTL
	\item Modificar lógica y analizar recursos
	\item Agregar una 5$^{ta}$ entrada adicional, repetir
\end{itemize}
\end{frame}

\end{document}
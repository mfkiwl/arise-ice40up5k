\documentclass{beamer}
\usetheme{Madrid}
\usecolortheme{default}

\definecolor{BlueITBA}{HTML}{00558c} 
\colorlet{beamer@blendedblue}{BlueITBA}
\setbeamertemplate{caption}{\scriptsize\insertcaption}
\setbeamertemplate{itemize items}[circle]
%\setbeamertemplate{itemize subitem}[triangle]
\setbeamertemplate{itemize subitem}{\raisebox{0.2em}{\scalebox{0.65}{$\blacktriangleright$}}}   % Second level

%\setbeamertemplate{enumerate items}[default]
\setbeamertemplate{section in toc}{\inserttocsectionnumber.~\inserttocsection}

\usepackage{sansmathfonts}				% Sans Serif equations
\usepackage[T1]{fontenc}					% Output font encoding for international characters
\usepackage[utf8]{inputenc}				% Encoding of files: utf8
\renewcommand*\familydefault{\sfdefault} 		% Sans Serif as default font
\usepackage{verbatim}
 
% Setup the language and its properties (choose only one)
\usepackage[spanish, es-tabla, es-nodecimaldot]{babel}
\addto\captionsspanish{\renewcommand{\contentsname}{Contenido}}

\usepackage{ulem}
\usepackage{graphicx}
\usepackage{tikz}						% Required for title page
\usepackage{hyperref}
\hypersetup{
    colorlinks=true,
    linkcolor=white,
    filecolor=magenta,      
    urlcolor=blue,
    citecolor=blue,    
}
\usepackage{listings}
\usepackage{adjustbox}
\definecolor{vgreen}{RGB}{104,180,104}
\definecolor{vblue}{RGB}{49,49,255}
\definecolor{vorange}{RGB}{255,143,102}
\lstset{
	basicstyle=\scriptsize,
	language=Verilog,
	tabsize=2,
	keywordstyle=\color{vblue},
	identifierstyle=\color{black},
	commentstyle=\color{vgreen},
	showstringspaces=false
}

\title[Primeros Pasos]{Primeros Pasos Apio}
\subtitle{}
\author{Dr. Ing. Pablo \textsc{Cossutta}}
\date{Octubre 2019}
\logo{\includegraphics[height=1.5cm]{../../logo_itba.png}} 
 
\begin{document}
\beamertemplatenavigationsymbolsempty % Disable Nav Bar
% Title and Intro
\maketitle
\logo{}

\begin{frame}{Tutorial}
¿Qué se va a hacer?
\begin{itemize}
	\item Instalar Python
	\begin{itemize}
		\item Seleccionar \texttt{pip}
	\end{itemize}
	\item Instalar Apio
	\item Crear un directorio de trabajo
	\item Cargar los archivos
	\item Synthesis, Map, Place \& Route, Bitstream
	\item Cargar el diseño
	\item Simular
\end{itemize}
¿Qué se necesita?
\begin{itemize}
	\item Archivo HDL (\texttt{gs\_apio.v})
	\item Constraints para implementación (\texttt{gs\_apio.pcf})	
	\item Testbench (\texttt{gs\_apio\_tb.pcf})	
\end{itemize}
\end{frame}
    
\begin{frame}[fragile]{gs\_apio.v - Definition and clock divider}
\begin{itemize}
	\item Similar al anterior solo modificar el nombre \texttt{rgb\_reg}
\end{itemize}
\begin{columns}
\column{0.5\textwidth}
\begin{center}\begin{adjustbox}{padding=2pt 0pt 0pt 0pt, cfbox=blue!50 1pt}\begin{lstlisting}
module top(
   output RGB0,
   output RGB1,
   output RGB2,
   input clk
);

localparam DIV_N = 22;

reg [DIV_N-1:0] div = 0;
always @(posedge clk) begin
	div <= div + 1'd1;
end

reg en;
always @(posedge clk) begin
	if (div==0) en <= 1'b1;
	else en <= 1'b0;
end
\end{lstlisting}\end{adjustbox}\end{center}
\column{0.5\textwidth}
\begin{center}\begin{adjustbox}{padding=2pt 0pt 0pt 0pt, cfbox=blue!50 1pt}\begin{lstlisting}
reg [2:0] rgb = 3'd0;
always @(posedge clk) begin
	if (en) begin
		case (rgb)
			3'b000:  rgb <= 3'b001;
			3'b001:  rgb <= 3'b010;
			3'b010:  rgb <= 3'b100;
			3'b100:  rgb <= 3'b000;
			default: rgb <= 3'b000;
		endcase
	end
end
\end{lstlisting}\end{adjustbox}\end{center}
\end{columns}
\end{frame}

\begin{frame}[fragile]{gs\_apio.v - SB\_RGBA Instantiation}
\begin{itemize}
	\item Cambia el nombre de la primitiva
\end{itemize}
\begin{center}\begin{adjustbox}{padding=2pt 0pt 0pt 0pt, cfbox=blue!50 1pt}\begin{lstlisting}
SB_RGBA_DRV RGBA_DRIVER (
	.CURREN(1'b1),
	.RGBLEDEN(1'b1),
	.RGB0PWM(rgb[0]),
	.RGB1PWM(rgb[1]),
	.RGB2PWM(rgb[2]),
	.RGB0(RGB0),
	.RGB1(RGB1),
	.RGB2(RGB2)
);

// Set parameters of RGBA_DRIVER
defparam RGBA_DRIVER.CURRENT_MODE = "0b0";
defparam RGBA_DRIVER.RGB0_CURRENT =  "0b000001"
defparam RGBA_DRIVER.RGB1_CURRENT =  "0b000001"
defparam RGBA_DRIVER.RGB2_CURRENT =  "0b000001"

endmodule
\end{lstlisting}\end{adjustbox}\end{center}
\end{frame}

\begin{frame}[fragile]{Restricciones (Constraints)}
Restricciones lógicas
\begin{itemize}
	\item No disponibles en Yosys
\end{itemize}
\vfill
Restricciones físicas
\begin{itemize}
	\item Necesarias para la implementación
	\item Hay que definir todo lo que sea E/S
\end{itemize}
\begin{block}{gs\_apio\_1.pcf}
\begin{lstlisting}
set_io clk 35
set_io RGB0 39
set_io RGB1 40
set_io RGB2 41
\end{lstlisting}
\end{block}
\end{frame}

\begin{frame}[fragile]{Testbench (analizar) \texttt{gs\_apio\_tb.v}}
\begin{center}\begin{adjustbox}{padding=2pt 0pt 0pt 0pt, cfbox=blue!50 1pt}\begin{lstlisting}
`default_nettype none
`timescale 1 ns / 100 ps

module led_cnt_tb();
parameter DURATION = 200;

reg clk;
always #1 clk = ~clk;

wire RGB0, RGB1, RGB2;
top top_inst (.RGB0(RGB0), .RGB1(RGB1), .RGB2(RGB2), .clk(clk));

initial begin
	$dumpfile("gs_apio_tb.vcd");
	$dumpvars(0, led_cnt_tb);
	#(DURATION) $display("End of simulation");
	$finish;
end

endmodule
\end{lstlisting}\end{adjustbox}\end{center}
\end{frame}

\begin{frame}[fragile]{Primitiva faltante}
\begin{itemize}
	\item Necesitamos la primitiva \texttt{SB\_RGBA\_DRV} para poder simular
\end{itemize}
\begin{center}\begin{adjustbox}{padding=2pt 0pt 0pt 0pt, cfbox=blue!50 1pt}\begin{lstlisting}
// Define primitive for simulation
module SB_RGBA_DRV #(
	parameter CURRENT_MODE=0,
	parameter RGB0_CURRENT=0,
	parameter RGB1_CURRENT=0,
	parameter RGB2_CURRENT=0
)(
	input CURREN,
	input RGBLEDEN,
	input RGB0PWM,
	input RGB1PWM,
	input RGB2PWM,
	output RGB0,
	output RGB1,
	output RGB2
);
assign RGB0 = (CURREN & RGBLEDEN) ? RGB0PWM : 1'b0;
assign RGB1 = (CURREN & RGBLEDEN) ? RGB1PWM : 1'b0;
assign RGB2 = (CURREN & RGBLEDEN) ? RGB2PWM : 1'b0;
endmodule
\end{lstlisting}\end{adjustbox}\end{center}
\end{frame}


\begin{frame}[fragile]{Testbench (corregido) \texttt{gs\_apio\_tb.v}}
\begin{center}\begin{adjustbox}{padding=2pt 0pt 0pt 0pt, cfbox=blue!50 1pt}\begin{lstlisting}
`default_nettype none
`timescale 1 ns / 100 ps

module top_tb();
parameter DURATION = 200;

reg clk = 0;
always #1 clk = ~clk;

wire RGB0, RGB1, RGB2;
top top_inst (.RGB0(RGB0), .RGB1(RGB1), .RGB2(RGB2), .clk(clk));

initial begin
	$dumpfile("gs_apio_tb.vcd");
	$dumpvars(0, top_tb);
	#(DURATION) $display("End of simulation");
	$finish;
end

endmodule
\end{lstlisting}\end{adjustbox}\end{center}
\end{frame}

\begin{frame}[fragile]{Trabajo en clase (Hands On)}
\begin{itemize}
	\item Registrar 2 veces la entrada
	\begin{itemize}	
		\item Prevenir metaestabilidad
	\end{itemize}
	\item Realizar el debouncing de una entrada correspondiente a un pulsador
	\begin{itemize}
		\item Considerando un debounce de N ciclos (ambas direcciones)
		\item Al detectar un pulso se enciende el led Rojo durante 500ms
	\end{itemize}
	\item Realizar un testbench completo
	\item Implementar y verificar experimentalmente
	\item Modificar para que no vuelva a encenderse si se mantiene el pulsador apretado
\end{itemize}
\end{frame}

\end{document}